\documentclass{lug}

\usepackage{fontawesome}
\usepackage{etoolbox}
\usepackage{etoolbox}
\usepackage{textcomp}
\usepackage[nodisplayskipstretch]{setspace}
\usepackage{xspace}

\newcommand{\textapprox}{\raisebox{0.5ex}{\texttildelow}}

\AtBeginEnvironment{minted}{\singlespacing\fontsize{10}{10}\selectfont}

\makeatletter
\makeatother

\title{Vim}
\author{Sumner Evans}
\institute{Mines Linux Users Group}

\begin{document}

\section{Text Editors}
\begin{frame}{Why do we need text editors when we have IDEs?}
    \begin{itemize}[<+->]
        \item Often need to edit configuration files and you don't want to fire
            up Visual Studio to edit a simple text file.
        \item Most tasks do not need any fancy IDE features to complete.  For
            example, writing a simple Markdown file, writing a \LaTeX\ document,
            writing this presentation with Beamer, programming, etc.
        \item Most of the time, users don't use nearly all of the features that
            IDEs provide. This is a waste of computer resources.
        \item Good text editors have the ability to be extended to bring
            IDE-like features to your text editor. This gives greater
            customisation ability to the user.
    \end{itemize}
\end{frame}

\begin{frame}{Why use a terminal-based text editor?}
    \begin{itemize}[<+->]
        \item Often need to edit files and you don't want to have to open a GUI
            file selector to navigate to the correct file and edit it. (In fact,
            there are some bad file selectors which don't even let you open
            dotfiles!)

        \item If you are SSH-d into a remote machine (for example, while
            administering a server), you may not have the luxury of a GUI (there
            is SSH tunnelling, but that is not always an option).

        \item These text editors are made by programmers for programmers.
    \end{itemize}
\end{frame}

\section{Vim}

\begin{frame}{Why use Vim over other terminal-based text editors?}
    \begin{itemize}[<+->]
        \item You don't need a mouse at all. In fact, by default, you
            \textit{can't} use your mouse.

        \item You can do everything with the keyboard in just a few keystrokes.

        \item The \texttt{CTRL} key is hard to reach on standard keyboards.

        \item But all of this is only possible because of \textbf{modal
            editing}.
    \end{itemize}
\end{frame}

\begin{frame}{Non Modal Editing}
    \begin{itemize}[<+->]
        \item In most editors, when you type with your keyboard, what you type
            is inserted right at the cursor location.
        \item To access functionality such as selection and scrolling you need
            to use your mouse.
        \item To access functionality such as find [and replace], you need to
            use a menu item or some complicated keyboard shortcut.
    \end{itemize}
\end{frame}

\begin{frame}{Modal Editing: The Ultimate Separation of Powers}
    \begin{itemize}
        \item Pressing the same keys do different actions depending on which
            mode you are in.
        \item This is extremely space-efficient.
    \end{itemize}
\end{frame}

\begin{frame}{Modes: Normal}
    \begin{itemize}
        \item This is the default (normal) mode.
        \item It is sometimes referred to as command mode.
        \item This is the mode that you are in when you enter Vim.
        \item You can tell you are in normal mode because there will be no text
            in the bottom left corner of your console window.
        \item Used to get to other modes, for cursor movement, copy/pasting,
            saving, etc\dots
        \item \textbf{To return here from other modes, press the \texttt{ESC}
            key.}
    \end{itemize}
\end{frame}

\begin{frame}{Modes: Insert}
    \begin{itemize}
        \item Allows you to actually type text.
        \item To get to insert mode from normal mode, press \texttt{i}.
    \end{itemize}
\end{frame}

\begin{frame}{Modes: Visual}
    \begin{itemize}
        \item Allows you to select text at a character-resolution and perform
            actions upon that selected text.
        \item To get to visual mode from normal mode, press \texttt{v}.
    \end{itemize}
\end{frame}

\begin{frame}{Modes: Visual-Line}
    \begin{itemize}
        \item Allows you to select text at a line-resolution and perform
            actions upon that selected text.
        \item To get to visual-line mode from normal mode, press \texttt{V}
            (capital V).
    \end{itemize}
\end{frame}

\begin{frame}{Modes: Visual Block Mode}
    \begin{itemize}
        \item Allows you to select text vertical blocks of text and perform
            actions on that selected text.
        \item To get to visual mode from normal mode, press \texttt{CTRL + v}.
    \end{itemize}
\end{frame}

\begin{frame}{Common Commands in Normal Mode}
    \begin{itemize}
        \item \texttt{h}, \texttt{j}, \texttt{k}, and \texttt{l} move the cursor
            left, down, up, and right, respectively.
        \item \texttt{i} puts you into insert mode, right where the cursor is.
        \item \texttt{I} puts you into insert mode at the beginning of the
            current line.
        \item \texttt{a} puts you into insert mode, one character to the right
            of the cursor.
        \item \texttt{A} puts you into insert mode at the end of the current
            line.
        \item \texttt{o} inserts a line below the current line, and puts you
            into insert mode on that line.
        \item \texttt{O} (capital O) is the same as lower-case \texttt{o}, but a
            line above.
    \end{itemize}
\end{frame}

\begin{frame}{Common Commands in Normal Mode}
    \begin{itemize}
        \item \texttt{dd} will delete the current line.
        \item \texttt{yy} will copy the current line.
        \item \texttt{cc} will delete the current line and put you into insert
            mode at the beginning of the line.
        \item \texttt{x} deletes the character under the cursor. \texttt{X}
            deletes the character before the cursor.
        \item \texttt{p} will paste whatever is currently in the paste buffer.
            \begin{itemize}
                \item How do you put something into the paste buffer? With
                    \texttt{x}, \texttt{dd}, or \texttt{yy}! These also function
                    as what you would think of as cut and copy.
                \item But I can't paste from other programs! Vim sucks. Use
                    \texttt{"+y} and \texttt{"+p} to copy and paste,
                    respectively, from the system clipboard. Alternatively, add
                    \texttt{set clipboard=unnamedplus} to your \texttt{.vimrc}.
            \end{itemize}
    \end{itemize}
\end{frame}

\begin{frame}{Common Commands in Normal Mode}
    \begin{itemize}
        \item \texttt{u} can be used to undo, and \texttt{Ctrl+r} to redo.
        \item \texttt{w} moves the cursor forward by one word at a time, and
            \texttt{b} moves it back.
        \item \texttt{gg} moves the cursor to the top of the file.
        \item \texttt{G} moves the cursor to the bottom of the file.
    \end{itemize}
\end{frame}

\begin{frame}{I'm Stuck and I can't exit Vim!}
    \begin{itemize}
        \item Type \texttt{:w} from normal mode to save the file (you can do
            this at any point in the edit process)
        \item \texttt{:q} will exit vim. If you have unsaved edits, it will warn
            you of this and not exit.
        \item \texttt{:q!} exits silently and without saving. Only use this if
            you really don't want your file changes!
        \item Lastly, these can be strung together to save and quit, i.e.
            \texttt{:wq}. There is also \texttt{:x}, which does the same thing
    \end{itemize}
\end{frame}

\begin{frame}{A Sample of Cool Other Commands}
    \begin{itemize}
        \item \texttt{ci(} ``change inside parentheses'' --- deletes everything
            inside the current parenthetical statement and then puts you into
            insert mode. Super useful for changing function parameters.
        \item \texttt{cap} ``change around paragraph'' --- deletes the paragraph
            and puts you into insert mode.
        \item \texttt{dw} ``delete word'' --- delete the next word.
        \item \texttt{d3w} ``delete word'' --- delete the next three words.
        \item \texttt{dt)} ``delet til )'' --- deletes everything on the line
            until the next \texttt{)}.
        \item \texttt{df)} deletes up to and including the next \texttt{)}.
        \item Lots more examples in Jack's talk from last year.
            (\url{https://github.com/jackrosenthal/lug-vim-awesome})
    \end{itemize}
\end{frame}

\begin{frame}{Plugins}
    There are a lot of plugins for Vim. I use 37 different plugins! If you want
    Vim to do something for you, Google it and someone has probably implemented
    that feature. (Or come to LUG for recommendations!)

    My configurations: \\
    \small\url{https://github.com/sumnerevans/dotfiles/tree/master/.vim}
\end{frame}

\begin{frame}{Installing Vim}
    \begin{itemize}
        \item Linux: install the \texttt{vim} package using your package manager
        \item macOS: install the \texttt{vim} using Homebrew
        \item Windows: Google it
    \end{itemize}
\end{frame}

\begin{frame}{References}
    I used Caleb Jhones' and Jack Rosenthal's presentations on Vim from last
    year as as inspiration/source code for this talk.
    \begin{itemize}
        \item Caleb's presentation: \url{https://github.com/ThirdOf5/VIM-Intro}
        \item Jack's presentation: \url{https://github.com/jackrosenthal/lug-vim-awesome}
    \end{itemize}
\end{frame}

\begin{frame}[standout]
    \Huge
    Questions?
\end{frame}

\end{document}
