\documentclass{lug}

\usepackage{fontawesome}
\usepackage{etoolbox}
\usepackage{etoolbox}
\usepackage{textcomp}
\usepackage[nodisplayskipstretch]{setspace}
\usepackage{xspace}

\newcommand{\textapprox}{\raisebox{0.5ex}{\texttildelow}}

\AtBeginEnvironment{minted}{\singlespacing\fontsize{10}{10}\selectfont}

\makeatletter
\makeatother

\title{Vim}
\author{Sumner Evans}
\institute{Mines Linux Users Group}

\begin{document}

\section{Text Editors}
\begin{frame}{Why do we need text editors when we have IDEs?}
    \begin{itemize}[<+->]
        \item Often need to edit configuration files and you don't want to fire
            up Visual Studio to edit a simple text file.
        \item Most tasks do not need any fancy IDE features to complete.  For
            example, writing a simple Markdown file, writing a \LaTeX\ document,
            writing this presentation with Beamer, programming, etc.
        \item Most of the time, users don't use nearly all of the features that
            IDEs provide. This is a waste of computer resources.
        \item Good text editors have the ability to be extended to bring
            IDE-like features to your text editor. This gives greater
            customisation ability to the user.
    \end{itemize}
\end{frame}

\begin{frame}{Why use a terminal-based text editor?}
    \begin{itemize}[<+->]
        \item Often need to edit files and you don't want to have to open a GUI
            file selector to navigate to the correct file and edit it. (In fact,
            there are some bad file selectors which don't even let you open
            dotfiles!)

        \item If you are SSH-d into a remote machine (for example, while
            administering a server), you may not have the luxury of a GUI (there
            is SSH tunnelling, but that is not always an option).

        \item These text editors are made by programmers for programmers.
    \end{itemize}
\end{frame}

\section{Vim}

\begin{frame}{Why use Vim over other terminal-based text editors?}
    \begin{itemize}[<+->]
        \item You don't need a mouse at all. In fact, by default, you
            \textit{can't} use your mouse.

        \item You can do everything with the keyboard in just a few keystrokes.

        \item The \texttt{CTRL} key is hard to reach on standard keyboards.

        \item But all of this is only possible because of \textbf{modal
            editing}.
    \end{itemize}
\end{frame}

\begin{frame}{Non Modal Editing}
    \begin{itemize}[<+->]
        \item In most editors, when you type with your keyboard, what you type
            is inserted right at the cursor location.
        \item To access functionality such as selection and scrolling you need
            to use your mouse.
        \item To access functionality such as find [and replace], you need to
            use a menu item or some complicated keyboard shortcut.
    \end{itemize}
\end{frame}

\begin{frame}{Modal Editing: The Ultimate Separation of Powers}
    \begin{itemize}
        \item Pressing the same keys do different actions depending on which
            mode you are in.
        \item This is extremely space-efficient.
    \end{itemize}
\end{frame}

\begin{frame}{Modes: Normal}
    \begin{itemize}
        \item This is the default (normal) mode.
        \item It is sometimes referred to as command mode.
        \item This is the mode that you are in when you enter Vim.
        \item You can tell you are in normal mode because there will be no text
            in the bottom left corner of your console window.
        \item Used to get to other modes, for cursor movement, copy/pasting,
            saving, etc\dots
        \item \textbf{To return here from other modes, press the \texttt{ESC}
            key.}
    \end{itemize}
\end{frame}

\begin{frame}{Modes: Insert}
    \begin{itemize}
        \item Allows you to actually type text.
        \item To get to insert mode from normal mode, press \texttt{i}
    \end{itemize}
\end{frame}

\begin{frame}{Modes: Visual}
    \begin{itemize}
        \item Allows you to select text and perform actions upon that selected
            text.
    \end{itemize}
\end{frame}

\begin{frame}{Installing Vim}
    \begin{itemize}
        \item Linux: install the \texttt{vim} package using your package manager
        \item macOS: install the \texttt{vim} using Homebrew
        \item Windows: Google it
    \end{itemize}
\end{frame}

\begin{frame}{Further Reading}
    \begin{itemize}
        \item My configurations: \url{https://github.com/sumnerevans/dotfiles}
    \end{itemize}
\end{frame}

\begin{frame}{References}
    I used Caleb Jhones' and Jack Rosenthal's presentations on Vim from last
    year as as inspiration (maybe bordering on plagiarism) for this talk.
    \begin{itemize}
        \item Caleb's presentation: \url{https://github.com/ThirdOf5/VIM-Intro}
        \item Jack's presentation: \url{https://github.com/jackrosenthal/lug-vim-awesome}
    \end{itemize}
\end{frame}

\begin{frame}[standout]
    \Huge
    Questions?
\end{frame}

\end{document}
